\chapter{Materia�y i metody}
\label{cha:materialy i metody}

%---------------------------------------------------------------------------

\section{Materia�y}
\label{sec:R3 materialy}

Zbi�r danych zosta� stworzony w 2017 roku poprzez przeprowadzenie ankiety "The National Survey on Drug Use and Health (NSDUH)" w�r�d obywateli Stan�w Zjednoczonych. Tw�rc� i patronatem tego corocznego przedsi�wzi�cia jest Departament Zdrowia i Opieki Spo�ecznej Stan�w Zjednoczonych. Ankieta zosta�a pierwszy raz przeprowadzona w 1971 roku i jest stale udoskonalana i dostosowywana do obecnych reali�w. Ka�dego roku dane te wykorzystywane s� do utworzenia raport�w i statystyk na temat popularno�ci u�ywek i ich nadu�ywania i wp�ywu na zdrowie i �ycie badanych \cite{NSDUH}. 

%---------------------------------------------------------------------------
\subsection{Wyb�r grupy ankietowanych}
\label{subsec:R3 wybor ankietowanych}
\cite{codebook}


%---------------------------------------------------------------------------
\subsection{Spos�b przeprowadzania ankiety}
\label{subsec:R3 przeprowadzanie ankiety}
Ankietowani odpowiadali anonimowo na zestaw przydzielonych im pyta�. 
%---------------------------------------------------------------------------

\subsection{Struktura danych}
\label{subsec:R3 struktura danych}

%---------------------------------------------------------------------------
\subsection{Grupy ryzyka}
\label{subsec:grupy ryzyka}

%---------------------------------------------------------------------------
\section{Zastosowane metody}
\label{subsec:R3 metody}


%---------------------------------------------------------------------------

\subsection{Wst�pne przetwarzanie sygna��w}
\label{subsec:R3 wstepne przetwarzanie sygnalow (normalizacja)}


%---------------------------------------------------------------------------

\subsection{Ekstrakcja cech}
\label{subsec:R3 ekstrakcja cech}


%---------------------------------------------------------------------------

\subsection{Selekcja cech}
\label{subsec:R3 selekcja cech}


%---------------------------------------------------------------------------

\subsection{Walidacja krzy�owa}
\label{subsec:R3 walidacja krzyzowa}


%---------------------------------------------------------------------------

\subsection{Metody uczenia maszynowego}
\label{subsec:R3 metody uczenia maszynowego}
Uczenie maszynowe, uczenie nadzorowane
Klasyfikaotory:
\begin{itemize}
	
	\item Las losowy
	\item Regresja logistyczna
	\item KNN
	\item SVN
\end{itemize}

%---------------------------------------------------------------------------

\subsection{Optymalizacja parametr�w}
\label{subsec:R3 optymalizacja parametrow}


%---------------------------------------------------------------------------

\subsection{Kryteria oceny wynik�w}
\label{subsec:R3 kryteria oceny wynikow}
